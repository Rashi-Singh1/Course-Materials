\documentclass{article}
\usepackage{multirow}
\usepackage[table]{xcolor}
\usepackage{tabu}
\usepackage[top=1.5in,bottom=1.5in,left=1.35in,right=1.35in]{geometry}
\begin{document}
	\title{Cost of fruits in India}
	\date{March 2017}
	\maketitle
	Food prices refer to the (averaged) price level for food in particular countries or regions or on a
	global scale. The food industry’s contribution to the price levels and fluctuations come from the food
	production process, food marketing and food distribution. Source of uncontrollable price fluctuations are
	varying crop yield from excess supply to harvest failure and food speculation activities. It is speculated
	that already the global climate change could be a major factor behind rising food prices.A continuing
	drought in South Africa may - amongst other factors - have food inflation soar 11%until end of 2016 ac-
	cording to the South African Reserve Bank.To a certain extent, adverse price trends can be counteracted
	by food politics. When food commodities become too expensive on the world market, food security is
	in danger especially for developing countries. In keeping with the supply and demand-principle, global
	prices will on average continue to rise with the growing world population.Food prices refer to the (av-
	eraged) price level for food in particular countries or regions or on a global scale. The food industry’s
	contribution to the price levels and fluctuations come from the food production process, food marketing
	and food distribution. Source of uncontrollable price fluctuations are varying crop yield from excess
	supply to harvest failure and food speculation activities. It is speculated that already the global climate
	change could be a major factor behind rising food prices.A continuing drought in South Africa may -
	amongst other factors - have food inflation soar 11\%until end of 2016 according to the South African
	Reserve Bank.To a certain extent, adverse price trends can be counteracted by food politics. When
	food commodities become too expensive on the world market, food security is in danger especially for
	developing countries. In keeping with the supply and demand-principle, global prices will on average
	continue to rise with the growing world population.\\
	\begin{table}[h]
		\centering
		\taburulecolor{blue}
		\renewcommand{\arraystretch}{1.5}
		\begin{tabular}{|p{2.5cm}|p{2.5cm}|p{2.5cm}|p{2.5cm}|p{2.5cm}|}
			\hline
			\rowcolor{yellow}\multicolumn{2}{|c|}{Fruit Details} & \multicolumn{3}{c|}{Cost Calculations} \\ \hline
			Fruit name & Type & No. of units & Cost/Unit & Cost(Rs.) \\ \hline
			\multirow[t]{2}{*}{Mango} & Malgoa & 20 & 75 & 1500 \\ \cline{2-5}
			& Alfonso & 20 & 85 & 1700 \\ \hline
			Jackfruits & Kolli Hills & \multirow[t]{2}{*}{10} & 50 & 500 \\ \cline{1-2} \cline{4-5} 
			Banana & Green & & 20 & 200 \\ \hline
			\multicolumn{4}{|c|}{\textbf{Total cost (Rs.)}} & \cellcolor{pink}\textbf{3900} \\ \hline
		\end{tabular}
		\color{red}\caption{\color{blue} Cost of fruits in India}
	\end{table}
	
\end{document}